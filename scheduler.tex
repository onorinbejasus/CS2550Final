The scheduler of \textit{myPTM} implements \textit{Strict Two-Phase Locking (S2PL)} as its lock manager. This sub manager works as follows:

\begin{algorithmic}

	\If{T wants to read/write on an object}
	\State Obtains S/X lock.
	\State Holds lock until end of transaction 
	\EndIf
   
\end{algorithmic}  . \\

This locking scheme thus guarantees serializability, a recoverable schedule, avoidance of cascading aborts, and that the precedence graph will be acyclic. $myPTM$ employs the following data structure to handle the locking:

\begin{itemize}
\item \textbf{lockTable}hash of lock entries: OID, Mode, List, List Wait queue
\subitem Number of transactions currently holding a lock
\subitem Type of lock held (shared or exclusive)
\subitem Pointer to queue of lock requests
\end{itemize}

The second submodule that the schedule controls is the deadlock detector. For this detection, \textit{myPTM} employs the \textit{Wait for Graph} deadlock detection scheme, where nodes of the graph are the transactions with edges from $node_i$ to $node_j$ represent a conflict of $node_i$ waiting for $node_j$. This submodule uses the data structure:

\begin{itemize}
\item \textbf{wfgMatrix} - keep track of dependencies 
\subitem TID
\subitem waitForTID
\end{itemize}

to keep track of the dependencies across the transactions. 

The following is how the module will be implemented. The variable \textbf{logFile} handles the scheduler's log: \\

\noindent\fbox{%
\begin{varwidth}{\dimexpr\linewidth-2\fboxsep-2\fboxrule\relax}
\begin{center}
\begin{algorithmic}[H]
\State \# Ensures strict 2PL and then passes command on to Data Manager
\State \#Return value passed back by Data Manager or blocked status 
\Function {handleCommand}{command, TID, dataItem} 
\State type = use command to determine which lock needed
\If{read/multiple read/write/delete}
	\If{checkLock(type,TID, dataItem)}
		\State pass on to Data Manager
	\Else
		\State lockStatus = reqLock(type, TID,dataItem)
		\If{lockStatus == failure}
			\State add to lockTable, wfgMatrix
			\Return blocked
		\Else
			\State pass on to Data Manager
		\EndIf
	\EndIf	
	
\ElsIf{commit or abort}
	\State pass on to Data Manager
	\State \textbf{releaseLocks(TID)}
\Else
	\State error, unknown command
\EndIf

\EndFunction
	
\State \# Release locks TID has acquired:
\State \# Approve a pending request
\State \# Update lockTable, wfgMatrix
\Function{releaseLocks}{TID}  \EndFunction
\State \# Return true if TID has lock of type on dataItem; otherwise, return false
\Function{checkLock}{type, TID, dataItem} \EndFunction
\State \# Attempt to acquire lock of type on dataItem; return true on success, else false
\Function {reqLock}{type, TID, dataItem} \EndFunction

\end{algorithmic}
\end{center}
\end{varwidth} % 
}

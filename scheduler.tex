The Scheduler has two submodules: Lock Manager and the Deadlock Detector. The Lock Manager implements \textit{Strict Two-Phase Locking (Strict 2PL)}. Locking is performed at the file level. The Lock Manager acquires the locks for a transaction or process on an as-needed basis. So, if a transaction/process T wants to read/write a file it will have to acquire the appropriate shared/exclusive lock. \\

The difference between a transaction and a process is releasing a lock. A transaction will hold all locks until commit or abort time when they are released all at once, whereas a process will release and acquire locks as needed on a per-command basis. Using Strict 2PL guarantees serializability, a recoverable schedule, avoidance of cascading aborts, and that the precedence graph will be acyclic. $myPTM$ employs the following data structure to handle the locking:

\begin{itemize}
\item \textbf{lockTable} - hash of lock entries: OID, Mode, List, List Wait queue
\subitem Number of transactions currently holding a lock
\subitem Type of lock held (shared or exclusive)
\subitem Pointer to queue of lock requests
\end{itemize}

The second submodule that the schedule controls is the Deadlock Detector. \textit{myPTM} employs the \textit{Wait-for Graph} deadlock detection scheme, where nodes of the graph are the transactions with edges from $node_i$ to $node_j$ represent a conflict of $node_i$ waiting for $node_j$. This submodule uses the following data structure (which is the adjacency matrix of the Wait-for Graph):

\begin{itemize}
\item \textbf{wfgMatrix} - keep track of dependencies 
\subitem TID
\subitem waitForTID
\end{itemize}

to keep track of the dependencies across the transactions. Deadlock detection will be performed periodically after a certain amount of time has passed. An arbitrary transaction involved in the deadlock will be selected to abort in order to break the deadlock. This submodule also has a variable \textbf{detectTime} that is the period of time used to wait between performing deadlock detection. This period is varied according to whether or not a deadlock is actually detected. \\

The following is how the Scheduler will be implemented. The variable \textbf{logFile} handles the scheduler's log: \\

\begin{mdframed}
\begin{algorithmic}[H]
\State \# Ensures strict 2PL and then passes command on to Data Manager
\State \#Return value passed back by Data Manager or blocked status 
\Function {handleCommand}{command, TID, dataItem} 
\State type = use command to determine which lock needed
\If{read/multiple read/write/delete}
	\If{checkLock(type,TID, dataItem)}
		\State pass on to Data Manager
	\Else
		\State lockStatus = reqLock(type, TID,dataItem)
		\If{lockStatus == failure}
			\State add to lockTable, wfgMatrix
			\Return blocked
		\Else
			\State pass on to Data Manager
		\EndIf
	\EndIf	
	
\ElsIf{commit or abort}
	\State pass on to Data Manager
	\State \textbf{releaseLocks(TID)}
\Else
	\State error, unknown command
\EndIf

\EndFunction

\State	
\State \# Update lockTable, wfgMatrix accordingly
\Function{releaseLocks}{TID} 
\For{each lock TID has} 
\State Release the lock
\State Approve a pending request (if any exist)
\EndFor
\EndFunction

\State
\State \# Return true if TID has lock of type on dataItem; otherwise, return false
\Function{checkLock}{type, TID, dataItem} \EndFunction

\State
\State \# Attempt to acquire lock of type on dataItem; return true on success, else false
\Function {reqLock}{type, TID, dataItem} \EndFunction

\State 
\State \# Use the wfgMatrix to detect deadlocks
\Function{detectDeadlock}{} 
\If {(deadlock detected)}
	\State TID = select arbitrary transaction involved in cycle
	\State Let TM know to abort the transaction and: \textbf{undoEffects}(TID) 
	\State Let DM know about aborted transaction
	\State \textbf{detectTime} = \textbf{detectTime} / 2
	\State \Return
\EndIf
\State \textbf{detectTime} = \textbf{detectTime} * 4	
\EndFunction

\end{algorithmic} 
\end{mdframed}

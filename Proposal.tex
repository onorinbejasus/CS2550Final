\title {
        \textbf{Pikachu's Revenge: Design Document} \\
}
\author {
Rebecca Hachey \\
Timothy Luciani \\
\textit{reh59@pitt.edu, tbl8@cs.pitt.edu}
}
\date {\today}

\documentclass[12pt]{article}

\newcommand{\tab}{\hspace*{3em}}

\usepackage{lingmacros}
\usepackage{tree-dvips}
\usepackage{graphicx}
\usepackage{fixltx2e}
\usepackage{verbatim}
\usepackage{amsmath}
\usepackage{framed}
\usepackage{algorithmic}
\usepackage{qtree}
\usepackage{tikz}
\usetikzlibrary{snakes}
\usepackage{fullpage}

\usepackage[paper=letterpaper,dvips,top=2.5cm,left=2.5cm,right=2.5cm,
    foot=2cm,bottom=2.5cm]{geometry}

\setlength{\parindent}{0in}

\begin{document}

 \maketitle

\section{Introduction} 

This design document presents the underlying features behind \textit{Pikachu's Revenge}'s \textit{myPTM}: my Pitt Transaction Manager. The goal of this system is to provide transactional limited support for a small database system by developing a concurrency control module, \textit{myPTM}. This control module will provide standard uncontrolled access to files while also providing both serialization and atomic access to the database using \textit{Strick Two-Phase Locking}. Our design aborts transactions due to deadlock while avoiding livelock all together by implementing the \textit{Wait for Graph} detection scheme. The following section outlines the overall design of the system.

\section{Design}

This section describes the implementation of \textit{myPTM}, including the data structures, variables, and functions to be implemented within each module.

\subsection{Main / Entry Point}

This module serves as the main entry point of the transaction manager. The purpose of this module is to process requests from the user, executing the appropriate modules depending on the request. This module is also responsible for keeping track of the global data structures \textbf{transManager}, \textbf{scheduler}, and \textbf{dataManager}; each one corresponding to the module of the same name. The definition of \textit{myPTM}:
\\
\begin{algorithmic}[1]

    \STATE \textbf{Inputs}: (Command Line)
    \STATE \tab{list of script files}
    \STATE \tab{bufferSize}
     \STATE \tab{seed = random number generator}
     \STATE \tab{numBufferPages}
     \STATE \tab{readMode (0 = round robin, 1 = random}
      \STATE \tab{searchMode (0 = scan, 1 = hash)}
      \STATE \textbf{Outputs}: (console)
       \STATE \tab{numCommitted = \# of committed transactions}
        \STATE \tab{percentRead = \% of read operations}
         \STATE \tab{percentWrite = \% of write operations}
    	 \STATE \tab{avgResponseTime = average response time}
	
	\STATE \textbf{ $main()$ } \COMMENT{Input and Output above}  \\
    
\end{algorithmic}  .\\

Studying the input and output above, the main module: takes in from the user the scripts; sets the buffer size and the number of buffer pages; initiates the random number see; and decides both the read mode and search mode. After execution of \textit{myPTM} is complete, the module outputs the number of committed transactions, the percent of both read and write operations, as well as the average response time of the entire execution, to the console or a log file for further analysis.

\subsection{ Transaction Manager }

\subsection{Scheduler}

\subsection{Data Manager}

\section{Implementation}

\section{Conclusion}

\end{document}